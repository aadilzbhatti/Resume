%% start of file `template.tex'.
%% Copyright 2006-2013 Xavier Danaux (xdanaux@gmail.com).
%
% This work may be distributed and/or modified under the
% conditions of the LaTeX Project Public License version 1.3c,
% available at http://www.latex-project.org/lppl/.


\documentclass[10pt,a4paper,sans]{moderncv}        % possible options include font size ('10pt', '11pt' and '12pt'), paper size ('a4paper', 'letterpaper', 'a5paper', 'legalpaper', 'executivepaper' and 'landscape') and font family ('sans' and 'roman')

% modern themes
\moderncvstyle{banking}                            % style options are 'casual' (default), 'classic', 'oldstyle' and 'banking'
\moderncvcolor{black}                                % color options 'blue' (default), 'orange', 'green', 'red', 'purple', 'grey' and 'black'
%\renewcommand{\familydefault}{\sfdefault}         % to set the default font; use '\sfdefault' for the default sans serif font, '\rmdefault' for the default roman one, or any tex font name
%\nopagenumbers{}                                  % uncomment to suppress automatic page numbering for CVs longer than one page

% character encoding
\usepackage[utf8]{inputenc}                       % if you are not using xelatex ou lualatex, replace by the encoding you are using
%\usepackage{CJKutf8}                              % if you need to use CJK to typeset your resume in Chinese, Japanese or Korean

% adjust the page margins
\usepackage[scale=0.85]{geometry}
%\setlength{\hintscolumnwidth}{3cm}                % if you want to change the width of the column with the dates
%\setlength{\makecvtitlenamewidth}{10cm}           % for the 'classic' style, if you want to force the width allocated to your name and avoid line breaks. be careful though, the length is normally calculated to avoid any overlap with your personal info; use this at your own typographical risks...

\usepackage{import}

% personal data
\name{Aadil}{Bhatti}
\address{1737 Primrose Ln, Glenview, IL, 60026}{}{}% optional, remove / comment the line if not wanted; the "postcode city" and and "country" arguments can be omitted or provided empty
\phone[mobile]{1 (847) 363-0768}                   % optional, remove / comment the line if not wanted
\email{abhatti2@illinois.edu}                               % optional, remove / comment the line if not wanted
\homepage{linkedin.com/in/aadilbhatti}                         % optional, remove / comment the line if not wanted
%\extrainfo{additional information}                 % optional, remove / comment the line if not wanted
%\photo[64pt][0.4pt]{picture}                       % optional, remove / comment the line if not wanted; '64pt' is the height the picture must be resized to, 0.4pt is the thickness of the frame around it (put it to 0pt for no frame) and 'picture' is the name of the picture file
%\quote{Some quote}                                 % optional, remove / comment the line if not wanted

% to show numerical labels in the bibliography (default is to show no labels); only useful if you make citations in your resume
%\makeatletter
%\renewcommand*{\bibliographyitemlabel}{\@biblabel{\arabic{enumiv}}}
%\makeatother
%\renewcommand*{\bibliographyitemlabel}{[\arabic{enumiv}]}% CONSIDER REPLACING THE ABOVE BY THIS

% bibliography with mutiple entries
%\usepackage{multibib}
%\newcites{book,misc}{{Books},{Others}}
%----------------------------------------------------------------------------------
%            content
%----------------------------------------------------------------------------------
\begin{document}
%\begin{CJK*}{UTF8}{gbsn}                          % to typeset your resume in Chinese using CJK
%-----       resume       ---------------------------------------------------------
\makecvtitle

\section{Education}

\vspace{6pt}
\cventry
{Expected May 2017}
{Mathematics \& Computer Science, Minor in Chemistry}
{University of Illinois at Urbana-Champaign}
{Champaign, IL}{}{Major GPA: 3.31}

\section{Computer Skills}
\vspace{5pt}
\begin{itemize}
  \vspace{3pt}
  \item
    \textbf{Languages}: Java, Python, C++, C, Haskell, Ruby, MATLAB
  \item
    \textbf{Web Development}: HTML, CSS, JavaScript, Django, Ruby on Rails, Node,
                              React, jQuery, D3
  \item
    \textbf{Mobile Development}: Android (Java, Android Studio), iOS (Swift, XCode)
  \item
    \textbf{Other Tools}: Git/GitHub, Amazon Web Services (EC2, S3, DynamoDB, Lambda),
    Arduino, Alexa Skills Kit (Amazon Echo)
\end{itemize}

\section{Work Experience}
\vspace{6pt}

\cventry
{August 2016--Present}
{\vspace{3pt}Undergraduate Teaching Assistant - CS 241}
{University of Illinois}
{Champaign, IL}{}
{
\begin{itemize}
  \item Course assistant for undergraduate System Programming course
  \item Developing programming assignments for students as well as course software
        infrastructure such as testing, continuous integration, and autograding of
        assignments
\end{itemize}}

\vspace{3pt}

\cventry
{May 2016--August 2016}
{\vspace{3pt}Software Engineering Intern}
{Capital One}
{Champaign, IL}{Information Security \& Risk Management (ISRM)}
{\vspace{3pt}
\begin{itemize}
  \item Built an SDK for iOS/Android apps which generates persistent
        Device ID for each user and associates with username / password.
        Used to match devices with users as part of multi-factor authentication support
  \item Learned about multi-factor authentication as well as development of iOS and Android
        applications
\end{itemize}}

\vspace{3pt}

\cventry
{February 2016--May 2016}
{\vspace{3pt}Software Engineering Intern}
{Capital One}
{Champaign, IL}{Cloud Migration Strategy}
{\vspace{3pt}
\begin{itemize}
  \item Determined the feasibility of using ``Juju Charms'' tool kit
        for deploying applications to Amazon Web Services without requiring extensive
        knowledge of AWS
  \item Built simple web application in Django and deployed with Juju
        as part of evaluation process
  \item Learned how to use various AWS services, app deployment and its roadblocks (scaling, security),
        using open-source software in enterprise environment
\end{itemize}}

\section{Projects}

\vspace{5pt}
\textbf{\href{http://cs-education.github.io/sys/}{Linux in the Browser}}
\begin{itemize}
  \item Developed under Professor Lawrence Angrave
  \item Worked on an in-browser System Programming IDE including C compiler \& terminal
  \item Responsible for making C compiler / terminal more portable for developers to use
        as well as developing Jupyter-like frontend as a use case of the IDE
  \item Created with Node.js, React.js, Jor1k
\end{itemize}

\vspace{3pt}
\textbf{\href{http://devpost.com/software/chimr}{Chimr}}
\begin{itemize}
  \item Built a ``Smart'' doorbell which alerts homeowner via SMS that someone is at door
  \item Won Best Use of Twilio API at WildHacks 2015
  \item Designed for use cases including notifying parents to let them know children home from school
  \item Created with Arduino, Python/Flask, Twilio API
\end{itemize}

\vspace{3pt}
\textbf{Rapsheet}
\begin{itemize}
  \item Built a full-stack web app to track \& analyze Twitter sentiment about Spotify albums
  \item Designed to search for trends between Twitter user opinions \& album popularity
  \item Created with Python/Django, Spotify API, Alchemy API, and D3.js
\end{itemize}

% Publications from a BibTeX file without multibib
%  for numerical labels: \renewcommand{\bibliographyitemlabel}{\@biblabel{\arabic{enumiv}}}% CONSIDER MERGING WITH PREAMBLE PART
%  to redefine the heading string ("Publications"): \renewcommand{\refname}{Articles}
% \nocite{*}
% \bibliographystyle{plain}
% \bibliography{publications}                        % 'publications' is the name of a BibTeX file

% Publications from a BibTeX file using the multibib package
%\section{Publications}
%\nocitebook{book1,book2}
%\bibliographystylebook{plain}
%\bibliographybook{publications}                   % 'publications' is the name of a BibTeX file
%\nocitemisc{misc1,misc2,misc3}
%\bibliographystylemisc{plain}
%\bibliographymisc{publications}                   % 'publications' is the name of a BibTeX file

%-----       letter       ---------------------------------------------------------

\end{document}

%% end of file `template.tex'.
